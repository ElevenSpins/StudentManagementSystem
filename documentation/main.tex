%888888 88     888888 Yb    dP 888888 88b 88 .dP"Y8 88""Yb 88 88b 88 .dP"Y8
%88__   88     88__    Yb  dP  88__   88Yb88 `Ybo." 88__dP 88 88Yb88 `Ybo."
%88""   88     88""     YbdP   88""   88 Y88 o.`Y8b 88"""  88 88 Y88 o.`Y8b
%888888 88ood8 888888    YP    888888 88  Y8 8bodP' 88     88 88  Y8 8bodP'
%
%
% Original author:
% Marian Mutschler (dev@mutschler-m.de) @ElevenSpins
%
%----------------------------------------------------------------------------------------
%	PACKAGES AND OTHER DOCUMENT CONFIGURATIONS
%----------------------------------------------------------------------------------------

\documentclass[a4paper, 11pt, oneside]{book} % A4 paper size, default 11pt font size and oneside for equal margins

\newcommand{\plogo}{\fbox{$\mathcal{PL}$}} % Generic dummy publisher logo

\usepackage[utf8x]{inputenc} % Required for inputting international characters
\usepackage[T1]{fontenc} % Output font encoding for international characters
\usepackage{fouriernc} % Use the New Century Schoolbook font
\usepackage{graphicx}
\usepackage{listings}
\usepackage[english, german]{babel} 



\usepackage{listings}
\usepackage{color}

%----------------------------------------------------------------------------------------
%	Code formating things
%----------------------------------------------------------------------------------------


\definecolor{mygreen}{rgb}{0,0.6,0}
\definecolor{mygray}{rgb}{0.5,0.5,0.5}
\definecolor{mymauve}{rgb}{0.58,0,0.82}

\lstset{ 
  backgroundcolor=\color{white},   % choose the background color; you must add \usepackage{color} or \usepackage{xcolor}; should come as last argument
  basicstyle=\footnotesize,        % the size of the fonts that are used for the code
  breakatwhitespace=false,         % sets if automatic breaks should only happen at whitespace
  breaklines=true,                 % sets automatic line breaking
  captionpos=b,                    % sets the caption-position to bottom
  commentstyle=\color{mygreen},    % comment style
  deletekeywords={...},            % if you want to delete keywords from the given language
  escapeinside={\%*}{*)},          % if you want to add LaTeX within your code
  extendedchars=true,              % lets you use non-ASCII characters; for 8-bits encodings only, does not work with UTF-8
  firstnumber=1,                % start line enumeration with line 1000
  %frame=single,	               % adds a frame around the code
  keepspaces=true,                 % keeps spaces in text, useful for keeping indentation of code (possibly needs columns=flexible)
  keywordstyle=\color{blue},       % keyword style
  language=Octave,                 % the language of the code
  morekeywords={*,...},            % if you want to add more keywords to the set
  numbers=left,                    % where to put the line-numbers; possible values are (none, left, right)
  numbersep=5pt,                   % how far the line-numbers are from the code
  numberstyle=\tiny\color{mygray}, % the style that is used for the line-numbers
  rulecolor=\color{black},         % if not set, the frame-color may be changed on line-breaks within not-black text (e.g. comments (green here))
  showspaces=false,                % show spaces everywhere adding particular underscores; it overrides 'showstringspaces'
  showstringspaces=false,          % underline spaces within strings only
  showtabs=false,                  % show tabs within strings adding particular underscores
  stepnumber=2,                    % the step between two line-numbers. If it's 1, each line will be numbered
  stringstyle=\color{mymauve},     % string literal style
  tabsize=2,	                   % sets default tabsize to 2 spaces
  title=\lstname                   % show the filename of files included with \lstinputlisting; also try caption instead of title
}

%----------------------------------------------------------------------------------------
%	TITLE PAGE
%----------------------------------------------------------------------------------------

\begin{document} 

\begin{titlepage} 
	\centering 
	\scshape 
	\vspace*{\baselineskip} 
	%------------------------------------------------
	%	Title
	%------------------------------------------------
	
	\rule{\textwidth}{1.6pt}\vspace*{-\baselineskip}\vspace*{2pt} 
	\rule{\textwidth}{0.4pt}
	
	\vspace{0.75\baselineskip} 
	
	{\LARGE Studnet Managemant System} % Title
	
	\vspace{0.75\baselineskip} 
	
	\rule{\textwidth}{0.4pt}\vspace*{-\baselineskip}\vspace{3.2pt} 
	\rule{\textwidth}{1.6pt} 
	
	\vspace{2\baselineskip} 
	
	%------------------------------------------------
	%	Subtitle
	%------------------------------------------------
	
	Finale Aufgabe % Subtitle / description
	
	\vspace*{3\baselineskip}
	
	%------------------------------------------------
	%	Editor(s)
	%------------------------------------------------
	
	Gruppenmitglieder
	
	\vspace{0.5\baselineskip} 
	
	{\scshape\Large Marcel Anders \\ Simon Baur \\ Marian Mutschler}
	
	\vspace{0.8\baselineskip}
	
	\textit{Programmieren in C \\ Frau Goertz}
	
	\vfill % 
	
	%------------------------------------------------
	%	Publisher
	%------------------------------------------------
	
    \includegraphics{assets/DHBW_MA_Logo.jpg}
	
	\vspace{0.3\baselineskip} 
	
	2021 

\end{titlepage}

%----------------------------------------------------------------------------------------

\tableofcontents
\newpage

\chapter{Allgemein}

\section{Allgmeines}
Im Rahmen des "Programmieren in C" Unterrichts an der DHBW Mannheim, sollten wir in Gruppen eine Aufgabe erledigen. Diese Gruppenarbeit hatte einen größeren Umfang um somit ein messbare Leistung der Studenten zu erheben. Aufgabe war es ein Programm zu entwickeln welches den Anforderungen in \textit{1.2 Intension / Aufgaenstellung} entspricht. \newline Für das Projekt wurde ein Deadline festgelegt. Innerhalb dieses Zeitraums müssen alle Anforderungen erfüllt sein.
\section{Intension / Aufgaenstellung} \label{aufgabe}
\begin{itemize}
\item Erstelle eine Studenten Struktur mit folgenden Inhalten: Nachname, Matrikelnummer, Start Datum, End Datum, geburstdatum
\item Erstelle eine Datums Struktur (day month year) fuer das gebursdatum und einschreibungsdatum etc.
\item Schreibe eine Funktion inputStudent: Bei dem der Benutzer alle relevanten Daten zu einem Studenten eingeben kann.
\item Schreibe eine Funktion addStudent: bei der ein Student hinzugefügt wird.
\item Schreibe eine Funktion in der die Anzahl der gespeicherten Studenten zurueck gegeben werden soll.
\item Schreibe eine Funktion printStudent(Matrikelnummer), mit der ein Student auf dem Bildschirm ausgegeben werden soll
\item Schreibe eine Funktion printAllStudents, mit der alle Studenten Alphabetisch auf dem Bildschirm ausgegeben werden sollen.
\item Schreibe eine Funktion menue, ueber die die Verschienen funktionen aufgerufen werden sollen.
\item Schreibe eine funktion deleteStudent(matrikelnummer), welche einen Studenten loescht.
\item Schreibe eine Funktion save die Alle gespeicherten Studenten in eine Datei speichert. Diese Funktion soll beim beenden des Programms automatisch aufgerufen werden und soll somit nicht im menue auftauchen
\item Schreibe eine Funktion read die eine Datei ausliest und alle Studenten daraus ins Programm laed. Diese Funktion soll beim start des Programms automatisch aufgerufen werden und soll somit nicht im menue auftauchen
\item Die Aufgabe soll mithilfe von Verketteten Listen gelöst werden.

\end{itemize}

\chapter{Student Management System} 
In diesem  Kapitel werden wir auf die einzelnen projektschritte eingehen. Jede verwendete bzw. entwickelte Funktion wird im folgenden erleutert. Die einzelnen Funktionen können in der "main.c" eingesehen werden. Diese liegt der Abgabe bei. \newline
This chapter contains each step of our project. Every functions, strucs or somethings else can be found in the main.c. This is only a discription of the function.
\section{Includes}
In diesem Schritt binden wir alle Bibliotheken ein. Wir haben verschiedene Hilfsbibliotheken eingbeunden. Im folgenden sind alle Bibliotheken zu sehen.
\begin{lstlisting}[language=C]
	#include <stdio.h>
	#include <stdlib.h>
	#include <string.h> //Damit strcpy() finktioniert
	#include <math.h> //Damit log() funktioniert
	#include <conio.h> //Damit getch() funktioniert
	#include <windows.h>	
\end{lstlisting}
\section{Umlaute}
Im folgenden werden wir alle Umlaute definieren damit diese in der Konsole dargestellt werden können. In diesem Fall haben wir die Umlaute "ß, ö, ä, ü, Ü" deklariert.
\begin{lstlisting}[language=C]
	#define sss 0xe1 
	#define oe 0x94 
	#define ae 0x84 
	#define ue 0x81 
	#define UE 0x9a
\end{lstlisting}

\section{Farben}
Im folgenden werden alle Umlaute deklariert, damit diese in der Konsole ausgegeben werden können.
\begin{lstlisting}[language=C]
	#define BLACK "\x1b[30m"
	#define RED "\033[0;31m"
	#define GREEN "\x1b[32m"
	#define YELLOW "\x1b[33m"
	#define BLUE "\x1b[34m"
	#define MAGENTA "\x1b[35m"
	#define CYAN "\x1b[36m"
	#define WHITE "\x1b[37m"
	#define RESET "\033[0m"
\end{lstlisting}
\subsection{Farbzuweisungen}
Hierbei ist zu beachten das ERR RED in der *.exe Anwendung nur mit <windows.h> funktioniert. Ebenfalls muss man system(color...) verwenden. Ansonsten ist dies nicht möglich.

\begin{lstlisting}[language=C]
	#define ERR RED
	#define SUCCESS GREEN
	#define IMPORTANTTEXT CYAN
\end{lstlisting}

\section{Struct Student}
 First we create the structure for our students
\begin{lstlisting}[language=C]
	struct student{
    	char *surname;
    	int matrikelnummer; //Muss 7 Zahlen lang sein
    	struct date startdate;
    	struct date exitdate;
    	struct date birthdate;
    	struct student *previous;
    	struct student *next;
}*start=NULL, *end=NULL;
\end{lstlisting}
\section{Sonstige Sonderzeichen}
Im folgenden werden die Elemente für eine saubere und auch schöne Darstellung der Tabellen in der Konsole definiert.
\begin{lstlisting}[language=C]
	#define MENU_ARROW 0x1a
	#define SPACE 0x20
	#define HORIZONLINE 0xcd
	#define VERTICALLINE 0xba
	#define CROSS 0xce
	#define CORNERDOWNLEFT 0xc8 
	#define CORNERDOWNRIGHT 0xbc 
	#define CORNERUPLEFT 0xc9 
	#define CORNERUPRIGHT 0xbb 
	#define TCROSSUP 0xca 
	#define TCROSSDOWN 0xcb 
	#define TCROSSRIGHT 0xcc 
	#define TCROSSLEFT 0xb9 
\end{lstlisting}

\section{Struct Date}
In diesem Schritt haben wir das Struct Date erstellt.
\begin{lstlisting}
	struct date{
        unsigned int day;
        unsigned int month;
        unsigned int year;
	};
\end{lstlisting}
\section{void wait}
Waiting for some user interaction.
\section{int getLength}
Checking the length of a integer. A simple example: 10 = 2, 100=3. Also it checks the integer isn't eqal 0 beacuse in this case we get some errors.

\section{checkDate}
Hier überprüfen wir den Userimput für das Datum. Das Ziel ist es dafür zu sorgen das es sich um ein echtes Datum handelt. Somit können fehler des Users ausgeschlossen werden.

\section{void addStudent()}
Diese Funktion bekommt lediglich Pointer übergeben. Die übergebenen Pointer haben die Aufgabe die Eingabe des User in die Liste hinzuzufügen. Im folgenden gehen wir auf signifikante Codebausteine ein:
\begin{lstlisting}[language=C]
	struct student *now, *before;
\end{lstlisting}
Dabei ist \textit{now} ein Pointer der auf den aktuellen Listeneintrag zeigt. Ebenfalls haben wir hier noch den Pointer \textit{before} der auf den Eintrag vor dem aktuellen Eintrag in der Liste zeigt.

\section{void inpitstudent(void)}
Diese Funktion erwartet eine \textit{void} übergabe. Die Funktion gibt dem User die Möglichkeit Alle Studentendaten einzugeben. Durch Eingabeeinschränkungen wird dafür gesorgt das der User möglichst wenig fehler macht. \newline Bei der Eingabe der Matrikelnummer wird überprüft ob diese sich bereits in der Liste befindet. Ebenfalls wird noch geprüft ob die Bedingung \textit{Geburtstag<Eintrittsdatum<Austrittsdatum} Zutrifft. Falls hierbei ein Wiederspruch vorliegt wird ein Fehler für den User Ausgeworfen.
\section{int countStudent(void)}
Hierbei wird die Anzahl der gespeicherten Studenten zurückgegeben.
\section{void printStudent(struct student *now)}
Hierbei wird die übergabe eines \textit{struct student} erwartet. Wenn dies der Fall ist wird der gewünschte Student mit der gesuchten Matrikelnummer ausgegeben.

\section{void deleteStudent(struct student *del)}
Diese Funktion hat die Aufgabe einen Studenten mit der gewünschten Matrikelnummer aus der Liste zu löschen. Hierbei wird der Funktion der Pointer des gewünschten Studenten übergeben. Damit wird sichergestellt das auch der richtige Student entfernt wird.

\section{struct student}
\begin{lstlisting}[language=C]
	struct student *merge(struct student *list1, struct student *list2){}
\end{lstlisting}
Hier benutzern wir die \textit{merge sort} in der \textit{top down} variante. Dazu bekommt man zwei Pointer auf den Start zweier Listen, dabei sollen diese beisen Listen sortiert und zu einer Liste zusammengesetzt werden. Ist das geschehen wird ein Pointer zurückgegeben, welcher auf den Start der neuen Liste zeigt.

\subsection{struct student // msort}
\begin{lstlisting}[language=C]
	struct student *msort(struct student *sort_start)
\end{lstlisting}
Bekommt einen Pointer übergeben. Ab diesem Pointer wird die Liste sortiert. Hierbei ist zu beachten, dass wenn der Pointer gleich \textit{NULL} ist oder die Liste auf die der Pointer verweist nur ein Eintrag ist, die Liste bereits Sortiert ist.
\begin{lstlisting}[language=C]
	if ((sort_start==NULL) || (sort_start->next==NULL)) return sort_start;
\end{lstlisting}
Folgend wird die Liste geteilt, damit man Sie sortieren kann. Daher suchen wir Sie mit einer \textit{for} Schleife ab. Dabei geht \textit{now} immer einen Eintrag weiter, währen \textit{after} immer zwei Einträge weiter geht, wenn z.B. \textit{after->next==NULL} ist wird die Schleife abgebrochen, weil after dann am letzen Eintrag angekommen ist.
\begin{lstlisting}[language=C]
	for (now=sort_start, after=sort_start->next; after && after->next; after=after->next->next) now=now->next;
\end{lstlisting}

\section{void sort(void)}
Hier wird die Liste sortiert. Dabei wird sie wie eine einfach verkettete Liste behandelt.
\section{void printAllStudents(void)}
Diese Funktion gibt alle Studenten in einer Liste aus und formatiert diese.

\section{void read(void)}
Hierbei handelt es sich um eine Kernfunktion der Anwendung. Diese Funktion ermöglicht es eine \textit{CSV} Datei einzulesen. Hierbei wird klassisch zwischen der ersten und der letzten zeile unterschieden. Die erste Zeile stellt die Headerzeile dar. Es wird immer eine ganze Zeile eingelesen. Jeder Teil der Zeile wird immer als \textit{String} eingelesen. Als trennzeichen wird das \textit{Komma} genommen. Ebenfalls wird der Strink wie in \textit{inputStudent()} auf seine richtigkeit überprüft.

\section{void save(void)}
Hier werden alle Studenten in der \textit{CSV} Datei gespeichert. Wenn erfolgreich gespeichert wurde wird der Speicher wieder mit dem Aufruf von \textit{free()} freigegeben.

\section{struct student *inputSearch(void)}
Diese Funktion lässt den Nutzer eine Matrikelnummer eigebe, es wird dann in der Liste gesucht ob es einen Eintrag gibt, es wird ein pointer auf das Element zurück gegeben, wenn eins gefunden wurde, NULL wenn die Liste leer ist oder nichts gefunden wurde.


\section{int menu(void)}
Hat die schlichte Aufgabe das Menü abzubilden.

\chapter{Main}
\begin{lstlisting}[language=C]
	int main(void){
    system("color"); 
    read();
    system("cls");
    int select;
    struct student *now;
    do{
        select=menu();
        switch(select){
        case 0:
        system("cls");
            printf("\t\t%c", CORNERUPLEFT); for(int i=1;i<=MENUMAX;i++) printf("%c", HORIZONLINE); printf("%c\n", CORNERUPRIGHT);
            inputStudent();
        break;
        case 1:
            system("cls");
            printf("\t\t%c", CORNERUPLEFT); for(int i=1;i<=MENUMAX;i++) printf("%c", HORIZONLINE); printf("%c\n", CORNERUPRIGHT);
            int len=3;
            int count=countStudent();
            len=4-getLength(count);
            printf("\t\t%c Es befinden sich " IMPORTANTTEXT "%d" RESET, VERTICALLINE, count);  printf(" Eintr%cge in der Datenbank!     ", ae); for(int i=0;i<len;i++) printf(" "); printf("%c\n", VERTICALLINE);
            printf("\t\t%c", CORNERDOWNLEFT); for(int i=1;i<=MENUMAX;i++) printf("%c", HORIZONLINE); printf("%c\n", CORNERDOWNRIGHT);
            wait();
        break;
        case 2:
            system("cls");
            printf("\t\t%c", CORNERUPLEFT); for(int i=1;i<=91;i++) printf("%c", HORIZONLINE); printf("%c\n", CORNERUPRIGHT);
            printf("\t\t%c Studenten suchen!\n", VERTICALLINE); 
            now=inputSearch();
            if(now){
                printStudent(now);
            }
            wait();
        break;
        case 3:
            system("cls");
            printAllStudents();
            wait();
        break;
        case 4:
            system("cls");
            printf("\t\t%c", CORNERUPLEFT); for(int i=1;i<=91;i++) printf("%c", HORIZONLINE); printf("%c\n", CORNERUPRIGHT);
            printf("\t\t%c Studenten l%cschen!\n", VERTICALLINE, oe); 
            now=inputSearch();
            if(now){
                deleteStudent(now);
            }
            wait();
            select=0;
        break;
        
        case 5:
        break;
        
        default:
            printf("\t\t%c " ERR "!Fehler select hat den ung%cltigen Wert '%d'!\n" RESET, VERTICALLINE, ue, select);
        break;
        }
    }while(select!=5);
    save();
    system("cls");
    return 0;
}
\end{lstlisting}
\end{document}